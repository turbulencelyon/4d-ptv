\documentclass{article}
\usepackage{siunitx}

\title{Pair dispersion toolbox}
\author{Thomas Basset}

\begin{document}

\maketitle

\section{Why to use it?}
For a given set of tracks, this MATLAB toolbox enables to compute pair dispersion statistics for positions and velocities.

\section{How to use it?}
As an example, we use the file \textit{tracks\_sample.mat} available in the HDF5 storage toolbox (cf. \textit{readme\_hdf5\_storage}). Once the distances $\Delta(t)$ and the velocity differences $\Delta v(t)$ for all pairs are computed, we can compute $\langle (\Delta(t) - \Delta_0)^2) \rangle = S_2(\Delta_0) t^2$ with $\Delta_0$ the initial separation (first order approximation). Because $S_2(\Delta_0) = \frac{11}{3} C_2 (\varepsilon \Delta_0)^{2/3}$ with $C_2 \simeq 2$ in homogeneous isotropic turbulence, we can extract $\varepsilon$ ($\varepsilon \simeq \SI{0.9}{W/kg}$). For the velocities, we can compute the autocorrelation function for $\Delta v(t)$ and extract the integral time scale $T$ then the parameter $\alpha$ defined as $\alpha = T / T_0$ with $T_0 = S_2(\Delta_0)/2\varepsilon$ (cf. Bourgoin \textit{JFM} 2015 dedicated to pair dispersion for more details, this $\alpha$ problem is under investigation). The script \textit{run\_pair\_dispersion} gives a run example.

\section{Functions}
\textit{help function name} gives some documentation, especially input and output arguments. These functions are commented and designed to be easily modified.
\begin{itemize}
\item \textit{lagstats\_onetrack}: compute Lagrangian statistics for one track
\item \textit{lagstats\_tracks}: compute Lagrangian statistics for all tracks with \textit{lagstats\_onetrack} and compute the mean over the tracks with \textit{meancell}
\item \textit{meancell}: compute the mean of array cells of different lengths
\item \textit{pairdisp\_proc}: compute $\Delta(t)$ and $\Delta v(t)$ for all pairs
\item \textit{pairdisp\_stats\_d}: compute $\langle (\Delta(t) - \Delta_0)^2) \rangle$ for a given range of $\Delta_0$
\item \textit{pairdisp\_stats\_v}: compute $\langle \Delta v(t+\tau) \Delta v(t) \rangle$ for a given range of $\Delta_0$
\end{itemize}

\end{document}